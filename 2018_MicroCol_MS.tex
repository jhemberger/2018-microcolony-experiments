\documentclass[11pt,]{article}
\usepackage[]{mathpazo}
\usepackage{amssymb,amsmath}
\usepackage{ifxetex,ifluatex}
\usepackage{fixltx2e} % provides \textsubscript
\ifnum 0\ifxetex 1\fi\ifluatex 1\fi=0 % if pdftex
  \usepackage[T1]{fontenc}
  \usepackage[utf8]{inputenc}
\else % if luatex or xelatex
  \ifxetex
    \usepackage{mathspec}
  \else
    \usepackage{fontspec}
  \fi
  \defaultfontfeatures{Ligatures=TeX,Scale=MatchLowercase}
\fi
% use upquote if available, for straight quotes in verbatim environments
\IfFileExists{upquote.sty}{\usepackage{upquote}}{}
% use microtype if available
\IfFileExists{microtype.sty}{%
\usepackage{microtype}
\UseMicrotypeSet[protrusion]{basicmath} % disable protrusion for tt fonts
}{}
\usepackage[margin=1in]{geometry}
\usepackage{hyperref}
\hypersetup{unicode=true,
            pdftitle={Saved by the pulse? Separating the effects of total and temporal food abundance on the performance of bumble bee microcolonies},
            pdfauthor={Jeremy Hemberger; Claudio Gratton; Agathe Frappa; Grant Witynski},
            pdfkeywords={\emph{Bombus impatiens}, colony growth, floral resources, temporal
variability, agroecosystems},
            pdfborder={0 0 0},
            breaklinks=true}
\urlstyle{same}  % don't use monospace font for urls
\usepackage{graphicx,grffile}
\makeatletter
\def\maxwidth{\ifdim\Gin@nat@width>\linewidth\linewidth\else\Gin@nat@width\fi}
\def\maxheight{\ifdim\Gin@nat@height>\textheight\textheight\else\Gin@nat@height\fi}
\makeatother
% Scale images if necessary, so that they will not overflow the page
% margins by default, and it is still possible to overwrite the defaults
% using explicit options in \includegraphics[width, height, ...]{}
\setkeys{Gin}{width=\maxwidth,height=\maxheight,keepaspectratio}
\IfFileExists{parskip.sty}{%
\usepackage{parskip}
}{% else
\setlength{\parindent}{0pt}
\setlength{\parskip}{6pt plus 2pt minus 1pt}
}
\setlength{\emergencystretch}{3em}  % prevent overfull lines
\providecommand{\tightlist}{%
  \setlength{\itemsep}{0pt}\setlength{\parskip}{0pt}}
\setcounter{secnumdepth}{0}
% Redefines (sub)paragraphs to behave more like sections
\ifx\paragraph\undefined\else
\let\oldparagraph\paragraph
\renewcommand{\paragraph}[1]{\oldparagraph{#1}\mbox{}}
\fi
\ifx\subparagraph\undefined\else
\let\oldsubparagraph\subparagraph
\renewcommand{\subparagraph}[1]{\oldsubparagraph{#1}\mbox{}}
\fi

%%% Use protect on footnotes to avoid problems with footnotes in titles
\let\rmarkdownfootnote\footnote%
\def\footnote{\protect\rmarkdownfootnote}

%%% Change title format to be more compact
\usepackage{titling}

% Create subtitle command for use in maketitle
\providecommand{\subtitle}[1]{
  \posttitle{
    \begin{center}\large#1\end{center}
    }
}

\setlength{\droptitle}{-2em}

  \title{Saved by the pulse? Separating the effects of total and temporal food
abundance on the performance of bumble bee microcolonies}
    \pretitle{\vspace{\droptitle}\centering\huge}
  \posttitle{\par}
    \author{Jeremy Hemberger \\ Claudio Gratton \\ Agathe Frappa \\ Grant Witynski}
    \preauthor{\centering\large\emph}
  \postauthor{\par}
      \predate{\centering\large\emph}
  \postdate{\par}
    \date{April 23, 2019}

\usepackage{setspace}

\doublespacing
\usepackage[left]{lineno}
\linenumbers

\begin{document}
\maketitle
\begin{abstract}
The loss of flower-rich habitat to agriculture is a key factor
contributing to bumble bee declines across Europe and North America.
Yet, agricultural intensification has not only altered flower abundance
in the landscape, but also affected when flowers are available during
the season (e.g., mass-flowering crops). While we know that both total
pollen and nectar and as well as temporal availability can impact bumble
bee colony success (growth and reproductive output), we have yet to
understand how these two factors combined might manifest. We designed an
experiment to decouple the effects of total food abundance and its
temporal availability on bumble bee microcolony development, by exposing
them to either constant or pulsed food availability at two ration
levels, 100\% and 60\% ad-libitum food abundance. Microcolonies provided
constant, full-rations of food grew the most, while those fed variable,
full-rations struggled to gain mass. Regardless of temporal
presentation, microcolonies fed 60\% ad-libitum rations gained little
mass over the experiment. Reproductive output was greatest in
microcolonies fed full-rations, regardless of the temporal availability
of food, while those given 60\% rations struggled to produce numerous
drones. This study highlights the importance of flower abundance in
agricultural landscapes for both colony growth and reproduction, and
suggests that increasing flower abundance could lead to improved colony
fitness.
\end{abstract}

\hypertarget{introduction}{%
\subsection{Introduction}\label{introduction}}

Bumble bee declines across Europe and North America are driven by a
number of interacting anthropogenic factors (e.g., pesticides (McArt et
al. 2017), novel diseases from managed bees (Brown et al. 2016; Fürst et
al. 2014) and climate change (Kerr et al. 2015)). There is a growing
consensus that habitat loss is the most important driver of decline
(Goulson and Nicholls 2016). In the US Midwest, two centuries of
agricultural intensification (defined here as the spatially extensive
increase of agrichemical use and proliferation of monocultures
(e.g.(Benton, Vickery, and Wilson 2003)) has removed bumble bee habitat
- supplanting once continuous landscapes of prairie, savanna, and
wetlands with highly productive agricultural crops (Rhemtulla,
Mladenoff, and Clayton 2007; Smith 1998). Coincident with the transition
to primarily agricultural land use in the US, several species of bumble
bee have declined precipitously (Grixti et al. 2009; Cameron et al.
2011, @Jacobson2018a). Indeed, threats posed to bumble bee populations
have arisen from both direct and indirect effects of agricultural
intensification. (Roulston and Goodell 2011).

Most importantly, agricultural intensification has led to wholesale
change in the abundance and temporal availability of floral resources in
the landscape (Goulson 2010; Schellhorn, Gagic, and Bommarco 2015;
Goulson et al. 2015; Vaudo et al. 2018). Historically, landscapes
containing a continuous supply of diverse floral resources (e.g.,
tall-grass prairies (Hines and Hendrix 2005)) were prolific, however
many of these landscapes have been lost to commodity agriculture (Smith
1998). Contemporary commodity crop landscapes in the Midwest, featuring
decreased crop diversity in combination with an increased use of
herbicides, have significantly reduced flowering plant availability
(Carvell et al. 2006; Goulson et al. 2015), creating proverbial food
deserts for bumble bees aside from small patches of remnant natural
habitat. In contrast, some agricultural landscapes contain
mass-flowering crops (e.g., fruit crops, canola) that provide large
pulses of floral resources, albeit over a short time period (Westphal,
Steffan-Dewenter, and Tscharntke 2009; Holzschuh et al. 2013; Rundlöf et
al. 2014). With respect to floral resource abundance, a range of
possible landscapes exist in agriculturally dominated regions. However,
the trend has moved toward highly simplified landscapes, with floral
resources available only during crop bloom.

The interaction between total floral resource abundance and temporal
availability in the landscape can be visualized conceptually, as two
potentially independent factors (Fig.1). In this abstraction, `Zone 1'
landscapes contain few flowers that are constantly available over the
course of the growing season. This can be contrasted with `Zone 3'
landscapes that contain an abundance of flowers, but primarily only
during two pulses (e.g., a mass-flowering crop). This framework allows
is to make simple predictions of how bumble bees are likely to respond
to changes in floral resource abundance in the landscape. We might
expect, given that bumble bees are obligate flower visitors existing in
long-lived colonies, that access do an abundance of flowers that are
consistently available over time (e.g., `Zone 2' landscapes) would be
critical for successful colony development.

Total floral resource abundance is an important factor that contributes
to the growth and reproductive success of bumble bee colonies. For
example, worker production is dependent on pollen and nectar influx to
the colony; shortfalls can be detrimental to worker output (Williams,
Regetz, and Kremen 2012), Carter and Dill, Sutcliffe and Plowright), and
reproductive (drone and gyne) output can be enhanced with increased food
availability (Pelletier and McNeil 2003). Despite variable effect sizes
and experimental methodologies, most studies tend to agree that
increased flower abundance leads to increased bumble bee abundance
(e.g., Carvell et al. 2007; Blaauw and Isaacs 2014) or colony
performance (e.g., Spiesman et al. 2017).

While total food availability is key for developing bumble bee colonies,
when food is available can be equally important (Schellhorn, Gagic, and
Bommarco 2015). Because bumble bee colonies have three distinct
bottlenecks wherein food availability is crucial to colony success
(queen colony establishment, colony worker buildup, and reproductive
production), food shortages during any period of the colony life cycle
can impair colony function. As such, continuous availability of flowers
in the landscape is believed to be paramount for bumble bees (e.g.,
Fig1, Zone 2)(Martins et al. 2018). Past work has examined the effect of
total food abundance (Rotheray, Osborne, and Goulson 2017) and temporal
availability (Schmid-Hempel and Schmid-Hempel 1998) on colony
development independently, however we have yet to test the interactive
effect of the two. Is a lack of food, the temporal availability of food,
or both limiting bumble bee colony success? Parsing these two
interacting factors apart could help explain specific mechanisms
underlying bumble bee declines in agricultural landscapes, as well as
provide conservation interventions.

To decouple the effects of total food abundance and temporal
availability on bumble bee colony development, we designed a 2x2
factorial experiment that varied total food amount and its temporal
availability. Treatment designs were meant to simulate hypothetical
scenarios that provide bumble bees either constantly available, or
variable food resources at two ration levels: 100\% and 60\% ad-libitum
(ad-lib) (Fig.1). These may represent different modalities of food
intake corresponding to different types of agricultural landscapes. For
example, bumble bees might experience landscapes containing few floral
resources and only during one short period of time (e.g., Fig1, Zone 4)
or a landscape containing many floral resources that are consistently
available (e.g., Fig1, Zone 2).

Using microcolonies of the Common Eastern bumble bee (\emph{Bombus
impatiens} Cresson), we tested the following hypotheses: (1)
microcolonies provided 100\% ad-lib food (pollen and nectar) in equally
sized rations would gain the most mass and have the greatest
reproductive output (e.g., zone 2); (2) microcolonies provided 100\%
ad-lib in pulses separated by periods of relative starvation would
perform slightly worse compared to those provided 100\% ad-lib in equal
rations, as \emph{B impatiens'} population stability in agricultural
landscapes suggests resilience to land-use change and variability in
food quantity (Cameron et al. 2011; Grab et al. 2019); (3) microcolonies
provided 60\% ad-lib would perform worst (zones 1 and 4), regardless of
temporal treatment (equal, ``zone 1'' vs.~pulse, ``zone 4'' rations), as
colonies would be too nutritionally stressed, regardless of relatively
large, episodic influxes of food in the pulsed colonies.

\hypertarget{materials-and-methods}{%
\subsection{Materials and methods}\label{materials-and-methods}}

\hypertarget{experimental-design-and-procedure}{%
\paragraph{Experimental design and
procedure}\label{experimental-design-and-procedure}}

To assess the impact of varying total, and temporal food abundance on
bumble bee colony production, we designed a 2x2 factorial experiment
utilizing microcolonies of \emph{B. impatiens}. Microcolonies were used
as proxies to full colonies given their ease of establishment and clear
parallels to full colony development (Dance, Botías, and Goulson 2017;
Moerman et al. 2017).

In three experimental rounds, we established microcolonies from: (Round
1) 10 queen-right colonies sourced from Koppert Biological Systems
(Howell, MI) in February of 2018; (Round 2) 10 queen-right colonies from
Koppert Biological Systems in May of 2018; (Round 3) 10 queen-right
colonies sourced from BioBest Biological (Romulus, MI) in October of
2018. In each round, microcolony initiation was identical. We removed
sets of 5 workers from a random colony and placed sets into 28
microcolony rearing boxes (10 x 15 x 10 cm - Sup Fig1), provisioning
each with 2 grams of honey bee collected pollen homogenized with nectar
and sealed in honey bee wax, as well as ad-libitum nectar through a
sub-floor cotton wick and reservoir (ProSweet: MannLake LTD, Minnesota).
To allow microcolony initiation, we left colonies undisturbed for
approximately 1 week. Once we observed evidence of microcolony
establishment (egg and larval brood cells) in each replicate box, we
removed any remaining pollen and nectar, and initiated treatment
regimes. In order to ensure microcolonies had equal capacity to respond
to food availability (i.e., 5 workers per box), we replaced any workers
throughout the experiment that died with randomly selected workers from
queen-right feeder colonies.

Over 8 weeks, we simulated landscape-scale food availability (both
pollen and nectar) in four treatments - each representing a hypothetical
landscape whereby total food abundance, and temporal availability were
independently altered according to the factorial design (Fig1).
Treatment conditions (hereafter, ``zones'') were as follows: ``Zone 1''
microcolonies were fed equal rations at 60\% ad-lib levels; ``Zone 2''
microcolonies were fed equal rations at 100\% ad lib levels; ``Zone 3''
microcolonies were fed in total the same amount of food as Zone 2
microcolonies, but food resources were provided as two large pulses with
periods of relative starvation (\textasciitilde{}60\% ad-lib)
encompassing the pulses; ``Zone 4'' microcolonies temporal availability
was the same as Zone 3, but with total food over the experiment equal to
Zone 1, and periods of relative starvation between pulses at
\textasciitilde{}38\% ad-lib. Seven microcolonies were randomly assigned
to each treatment. To determine total food rations supplied in each
treatment over the course of the experiment, we used measurements
reported in (Rotheray, Osborne, and Goulson 2017) (which reported on
ad-lib consumption rates of the ecological analog of \emph{B.
impatiens}, \emph{B. terrestris}), scaling total experiment food
abundance at 100\% ad-lib to 5 workers (approximately 33 grams of pollen
and 300 grams of nectar) over 17 feeding intervals (\textasciitilde{} 8
weeks). Reduced-ration treatments were scaled to 60\% of that value.

Every 3 days (hereafter interval feeding days: IFD), we massed whole
microcolonies by placing them on a scale and recording mass to the
nearest 0.01 grams. After massing, we fed microcolonies by providing an
appropriately massed pollen ball and nectar cup (see Supplemental
Material for feeding schedule and layout of microcolony box) depending
on the food treatment. After every 3 IFDs (every 9 Julian days), we
removed and massed the pollen that was not consumed in the microcolony
to calculate consumption. Nectar rations were replaced at every IFD -
each massed before addition and after removal to determine nectar
consumption (similar to (Rotheray, Osborne, and Goulson 2017)). We
massed the microcolony before the removal or addition of food to ensure
that measurements were comparable between IFDs.

To determine if treatments affected the relative fitness of each
microcolony, we censused colonies at each IFD. Censusing included
tallying worker mortality, drone (male) emergence, and presence of
different brood stages within the microcolony. After tallying drones
that had emerged, we removed and froze them for subsequent analysis. We
determined average drone wet mass per IFD for each microcolony,
individual drone intertegular distance (to nearest 0.01mm using ProgRes
CapturePro v2.0), and average drone fat content per IFD (\% fat in dry
mass) using ether extraction method described by (Samuelson et al.
2018)).

\hypertarget{data-analysis}{%
\paragraph{Data analysis}\label{data-analysis}}

We performed all data management and statistical analyses in R, version
3.5.1 ({\textbf{???}}). For each colony, we calculated the estimated
``actual'' microcolony mass relative to IFD 1. The goal of this
calculation was to best determine how much biomass was being added to
the actual brood mass within the microcolony between IFDs while
factoring out added, but not consumed, food:

\[
\begin{aligned}
Estimated Brood Mass_{IFD = n} = Microcolony Mass_{IFD = n} - Microcolony Mass_{IFD = 1} + \\
Pollen Consumed_{IFD = n} - Pollen Added_{IFD = n}
\end{aligned}
\]

where Microcolony Mass\textasciitilde{}IFD is the mass of the entire
microcolony (including box) at \texttt{IFD\ =\ n};
\(Microcolony Mass_{IFD = 1}\) is the mass of the entire microcolony on
the first IFD: we calculated mass gains relative the mass at IFD = 1;
\(Pollen Consumed_{IFD = n}\) is the average pollen consumed at
\texttt{IFD\ =\ n}, determined by taking the pollen consumed over the
course of 3 IFDs (determined after removing unconsumed pollen every 3
IFDs) and dividing by 3; and \(Pollen Added_{IFD = n}\) is the mass of
pollen added at at \texttt{IFD\ =\ n} (see example calculation in
Supplemental Material). In the event that missing data prevented
\(Estimated Brood Mass_{IFD = n}\) from being determined (\textless{}
1\% of data), we interpolated the missing values using the
\texttt{na.approx} function from the \texttt{zoo} R package. Nectar
storage/consumption within the microcolony was not considered for the
\(Microcolony Mass_{IFD = n}\) calculation, as we were not able to parse
out nectar consumed by workers from nectar moved by workers to honey
pots in the brood mass. At the end of the experiment, we also measured
the mass of the brood cluster (including all wax, remaining pupae,
larvae, eggs, and nectar) alone after removing them from the microcolony
boxes.

To evaluate whether treatments affected end of experiment brood mass,
drone production, drone fitness parameters (wet mass and IT distance),
or worker mortality, we constructed linear mixed-effects models (LMMs)
on the combined data set from all three experimental rounds. Each model
took the general form of a given response variable as a function of
treatment (full-factorial between total food abundance and temporal
variability) and experimental round, with random intercept and slope
estimates for each microcolony. We estimated treatment means from LMMs
(package: \texttt{lsmeans}) and used Tukey corrected pair-wise
comparisons (package: \texttt{multcomp}) to determine significance
between treatments. All end-of-experiment total LMMs were built using
the \texttt{nlme} package.

We also constructed repeated measures ANOVAs using the \texttt{nlme}
package to model colony mass, IFD average and cumulative drone
production, IFD average and cumulative worker mortality, as well as IFD
average and cumulative nectar and pollen consumption throughout the
course of the 8-week experiments. These models took the general form of
a given response as a function of treatment, date, and round, with a
random effect of colony identity. To account for temporal
autocorrelation, each analysis included a first order autocorrelation
structure (function: \texttt{corAR1}) with a time covariate of
measurement date and a grouping factor of colony identity. We specified
the autocorrelation correction using the lag = 1 value from an identical
model fitted with no autocorrelation structure.

Lastly, we calculated the growth rate of each microcolony for time
periods relative food pulses (Before pulse 1, during pulse 1, after
pulse 1 and before pulse 2, during pulse 2, and after pulse2). This was
accomplished by fitting a linear model of microcolony mass as a function
of IFD (time). We then extracted each slope coefficient (i.e., the
growth rate for a given microcolony during an aforementioned time
period) and constructed an ANOVA for each time period to determine
differences in growth rates among treatments, within time periods. The
\emph{P}-values associated with our initial slope estimates (mass as a
function of IFD) were used to test whether growth was different than
zero for a given time period. We also constructed a repeated measures
ANOVA for all slope estimates across all time periods to determine if
there were statistical differences in temporal microcolony growth rates.

\hypertarget{results}{%
\subsection{Results}\label{results}}

\hypertarget{microcolony-establishment}{%
\paragraph{Microcolony establishment}\label{microcolony-establishment}}

Overall, microcolony establishment success was high across experimental
rounds, with only 3 of 84 failing to initiate (96\% success). Two
additional microcolonies contained a hyper-aggressive worker that killed
all original and replacement nest-mates when establishing dominance.
Despite increased aggression, these single-worker colonies still
produced males. However, we removed them from analyses as the response
of a single-worker microcolony to treatments was not comparable to
standard, five-worker microcolonies.

\hypertarget{microcolony-growth-mass-and-food-consumption}{%
\paragraph{Microcolony growth (mass and food
consumption)}\label{microcolony-growth-mass-and-food-consumption}}

The magnitude, and pattern of microcolony growth depended strongly on
treatment throughout the course fo the experiment (Fig 2A,C:
F\textsubscript{3,73} = 33.36, \emph{P} \textless{} 0.001). End of
experiment microcolony mass was driven by an interaction of both total
food abundance, as well as temporal availability (Fig 2B,D: interaction
effect, F\textsubscript{1,24} = 6.72, \emph{P} = 0.016). In the constant
ration treatments (zones 1 and 2), end of experiment mass for zone 2
microcolonies was, on average, 92\% greater than zone 1 microcolonies
(Fig 2B, t\textsubscript{24} = 4.35, \emph{P} = 0.001). In the pulsed
ration treatments (zones 3 and 4), end of experiment mass for zone 3
microcolonies was, on average, 27\% more than zone 4 microcolonies,
however the difference was not statistically clear (Fig 2D,
t\textsubscript{24} = 1.53 , \emph{P} = 0 .437). Average mass at the end
of the experiment for the two 100\% ad-lib treatments (zones 2 and 3)
was 56\% greater for zone 2 relative to zone 3 (t = 3.523, \emph{P} =
0.008). There was a statistically clear effect of experimental round on
end of experiment mass, with rounds 2 and 3 overall weighing less at the
end of the experiment relative to round 1 (F\textsubscript{2,50} =
27.37, \emph{P} \textless{} 0.001). However, the pattern of end of
experiment mass across treatments was consistent between rounds
(F\textsubscript{6,43} = 0.685, \emph{P} = 0.663).

Microcolony growth rates over the course of the experiment varied by
treatment (Fig 3: F\textsubscript{3,24} = 5.60, \emph{P} = 0.004).
Overall, zone 2 average growth rate was the highest (t\textsubscript{24}
= 3.79, \emph{P} = 0.004), while zone 1 growth rates were only
statistically above 0 during one time period (before pulse 1). Growth
rates were similar among pulsed-ration treatments. Both pulsed-ration
treatments experienced negative growth during periods following food
pulses (after pulse 1/before pulse2, after pulse 2). While mass gain was
negligible during the first pulse, it was greatest during the second
food pulse. Surprisingly, there was no clear difference in growth rates
between zones 2 (0.702 +/- 0.111 grams per period) and 4 (0.353 +/-
0.111 grams per period), despite having different end of experiment
brood and average estimated masses.

Pollen consumption was impacted by both total food abundance and
temporal availability (F\textsubscript{1,24} = 6.07, \emph{P} = 0.021),
with constant 100\% ration microcolonies consuming the most (Pollen:
14.80 +/- 0.813 grams). Zone 3 microcolony pollen consumption was not
significantly different than zones 1 and 4, despite a 40\% difference in
pollen availability over the course of the experiment. Nectar
consumption followed a similar pattern, however consumption was driven
by total abundance, and not temporal availability (F\textsubscript{1,24}
= 44.24, \emph{P} \textless{} 0.001), with zones 2 and 3 consuming a
similar amount of nectar (119.10 +/- 3.60 grams and 112.26 +/- 3.53
grams, respectively: Supplemental Fig2).

\hypertarget{microcolony-demography-drone-production-and-worker-mortality}{%
\paragraph{Microcolony demography (drone production and worker
mortality)}\label{microcolony-demography-drone-production-and-worker-mortality}}

We found that total drone production varied as a function of total food
abundance (Fig 4: F\textsubscript{1,24} = 15.34, \emph{P} \textless{}
0.001). Zone 2 microcolonies produced numerically the most males, but
there was not a statistically clear difference between zones 2 and 3
(t\textsubscript{24} = 1.538, \emph{P} = 0.431). Both 60\% ad-lib
treatments produced on average 27\% fewer drones, regardless of the
pulse treatment. There was an effect of experimental round on drone
production, with rounds 2 and 3 overall producing fewer drones than in
round 1. However, like with end of experiment mass, the pattern of drone
production relative to treatments was similar across experimental rounds
(F\textsubscript{6,43} = 0.685, \emph{P} = 0.663). Worker mortality was
identical regardless of treatment, with on average 9 worker deaths per
microcolony throughout the course of the 8 week experiment (average of
1.1 per week).

The efficiency of conversion from food to drone was roughly similar
across treatments: to produce 1 drone required on average 0.70 +/- 0.08
grams of pollen and 10 +/- 2.54 grams of nectar. While there were no
clear statistical differences (treatment ration size effect:
F\textsubscript{1,24} = 0.189, \emph{P} = 0.667; treatment pulse effect:
F\textsubscript{1,24} = 0.018, \emph{P} = 0.893), zones 2 and 4
microcolonies were numerically more efficient in producing drones,
requiring on average 18\% less pollen to produce a single drone. Zones 1
and 3 were the least efficient, especially with nectar, 56\% more to
produce a single male relative to zones 2 and 4, however the differences
were not statistically clear.

\hypertarget{discussion}{%
\subsection{Discussion}\label{discussion}}

By manipulating the amount and temporal availability of pollen and
nectar, we show that both temporal availability and total food amount
affect microcolony growth and reproduction, respectively. End of
experiment microcolony mass was greatest when microcolonies were
provided constant, full-rations. This pattern matched our prediction
that bumble bee microcolonies would grow the most when provided
resources that mimic landscapes containing a high abundance of
temporally consistent resources. Indeed, microcolonies provided reduced,
or temporally variable rations struggled to gain mass, with several
exhibiting a net loss over the experiment. However, drone production,
arguably the most important metric of microcolony success, was only
impacted by total food amount: colonies fed full rations produced more
drones, regardless of temporal availability (i.e., Zone 2 vs.~Zone 3).
The contrasting effects of total (e.g., Rotheray, Osborne, and Goulson
2017) and temporal (e.g., Schmid-Hempel and Schmid-Hempel 1998) food
availability demonstrated in this study suggest that both factors are
important to the overall success of \emph{B. impatiens} microcolonies.

\hypertarget{microcolony-mass-gain}{%
\paragraph{Microcolony mass gain}\label{microcolony-mass-gain}}

Microcolonies provided constant, full-rations of food consistently
gained mass over the course of the experiment. Regardless of their
magnitude, food pulses were unable to rescue microcolony mass from
dearth periods. In fact, microcolonies experiencing pulsed food
availability exhibited dramatic swings in mass gain and loss coincident
with pulse and dearth periods, respectively. In contrast, microcolonies
experiencing constant rations were more consistent in their mass gain,
with full-ration treatments on average gaining mass across all
experimental periods, and reduced-ration treatments functionally
remaining at zero.

Many studies examining bumble bee colony responses to environmental
variables use mass as a proxy for reproductive output (Goulson et al.
2002; Elliott 2009; Westphal, Steffan-Dewenter, and Tscharntke 2009;
Williams, Regetz, and Kremen 2012). Indeed, mass in our experiments did
tend to correlate with increased reproductive output (especially in the
case of zone 2). In the absence of demographic data, the average lower
final mass of zone 3 microcolonies might have suggested lower colony
reproductive output. However, drone production was equivalent in pulsed,
full-ration colonies - a signal that colony mass alone may not tell the
complete story of bumble bee colony success. This finding corroborates
other studies examining bumble bee reproductive success (Williams,
Regetz, and Kremen 2012), and highlights the importance of additional
supporting evidence to accompany trends in colony mass gain.

For our experiment, microcolony mass was our best estimate of the total
biomass of workers, brood, and nesting materials at any given timepoint.
Because bumble bees store nectar within their brood cluster, our
estimate of microcolony mass may more accurately capture nectar
acquisition and depletion. If so, this metric is still important as
stored nectar is critical for brood incubation and worker caloric intake
(Heinrich 2004; Goulson 2010, @Rotheray2017). Additionally, estimated
average microcolony mass at the last experimental time point corresponds
well to the average final brood mass measured after microcolony
termination (R\textsuperscript{2} = 0.59), suggesting our calculation of
microcolony mass is accurate. Regardless of its true constituents,
colony mass is best used in tandem with demographic or additional
physical characteristics of the colony (e.g.~estimated colony volume) in
inferring colony ``success''. Patterns of mass gain/loss are likely more
appropriate to describe relative food intake and consumption, rather
than reproductive output. Despite this, colony mass is still an
important metric given that size is an important precursor to the
reproductive switch point of the colony (Goulson 2010).

\hypertarget{drone-production}{%
\paragraph{Drone production}\label{drone-production}}

A lack of an interaction between total food and temporal availability
revealed a statistically clear, positive effect of increased food
abundance on drone production independent of temporal availability. This
result supports our hypothesis regarding \emph{B. impatiens} suspected
tolerance of highly variable food environments. That is, over time for
\emph{B.impatiens}, when food is presented is less important to drone
productivity than how much food is available. In fact, populations of
\emph{B. impatiens}, among several other species, remain stable in
agriculturally dominated landscapes despite dearth periods of food,
while other species (namely \emph{B. affinis} and \emph{B. terricola})
have declined (Cameron et al. 2011). Worker polymorphism within \emph{B.
impatiens} could explain this tolerance, as smaller workers are more
robust to periods of nectar starvation (Couvillon and Dornhaus 2010). In
our study, however, worker body size was not controlled for, as workers
selected for a given microcolony were selected at random from natal
colonies. Worker mortality was consistent across microcolonies and
treatments, suggesting we did not unintentionally select smaller, more
tolerant workers for any given treatment.

Drone size was greatest under full-ration conditions. This is an
important difference, as drones are crucial for gene flow via dispersal.
Therefore, colonies producing relatively large-bodied drones (which
correlates with flight range (Greenleaf et al. 2007)) may be more
successful in contributing genetic information to subsequent
generations. We might also expect that the total food-driven difference
in drone size that we observed in this study would translate to
full-colonies producing workers rather than drones. Larger workers are
known to have colony-level benefits thanks to increased foraging range
and efficiency (Peat, Tucker, and Goulson 2005; Rotheray, Osborne, and
Goulson 2017). Interestingly, the treatment that produced the largest
drones, Zone 3, was one of the least efficient at converting pollen into
drones (though the difference in pollen consumption efficiency was not
statistically clear). This suggests that, while \emph{B. impatiens}
seems robust to temporal fluctuations in food abundance, it comes at a
cost of efficiency of food use.

Environmental stressors like food abundance and temporal availability
are likely to impact species differently (Roulston and Goodell 2011;
Woodard and Jha 2017). While studies examining \emph{B. impatiens}
reproductive response to variable food abundance are lacking,
experiments have documented enhanced reproductive success in variable
resource environments for the European ecological analog, \emph{B.
terrestris} (Schmid-Hempel and Schmid-Hempel 1998). For example,
(Schmid-Hempel and Schmid-Hempel 1998) found that variable access to
food led to an increased rate of food collection (mass gain) and
increased production of workers. We found similar patterns of increased
food collection rate among \emph{B. impatiens} microcolonies fed
variable rations, especially during the second pulse of our experiment.
However, we did not find variable food abundance elevated drone
production, suggesting that bumble bee responses to total food abundance
and temporal availability are likely to be species-specific. Given this,
it is important for future studies to consider comparative, interspecies
studies which could identify species most sensitive to temporal
bottlenecks in food availability (Schellhorn, Gagic, and Bommarco 2015).
Such work would build on findings highlighting the importance of
resource continuity for wild bee communities (Martins et al. 2018), and
could aid in the design of more friendly agricultural landscapes (Landis
2017).

\hypertarget{conclusions}{%
\paragraph{Conclusions}\label{conclusions}}

Disentangling the contribution of environmental stressors to bumble bee
decline is imperative if we are to be successful in stemming further
declines. In this study, we showed that temporal and total pollen and
nectar availability interact to impact bumble bee microcolony growth,
with microcolonies provided more, temporally consistent food growing the
most. We also showed that reproductive output was driven by total, and
not temporal pollen and nectar availability: microcolonies provided full
rations of pollen and nectar produced the most drones. While we examined
laboratory microcolonies, the responses observed should be indicative of
a standard, queen-right colony (Tasei and Aupinel 2008; Dance, Botías,
and Goulson 2017). If anything, free-foraging, queen-right colonies are
likely to see exacerbated responses to similar treatments, given that
foraging is the most energetically expensive and risky activity for
bumble bees. Even though temporal availability did not impact the
reproductive output of \emph{B. impatiens} in this study, other species
of bumble bee need to be examined for their ability to cope with
boom-bust cycles of food availability. The implications of this work to
managing landscapes is clear: at the least, an increase in total floral
abundance (i.e., pollen and nectar) would likely have benefits to bumble
bee colony reproduction. However, increasing both total floral abundance
as well as temporal continuity would benefit species tolerant of dearth
periods, as well as those sensitive to nutritional stress. Such efforts
are essential to limit further loss of essential ecosystem service
providers like bumble bees.

\begin{figure}
\centering
\includegraphics{./fig1_conceptual.png}
\caption{\textbf{Figure 1:} Microcolony landscape and treatment
concepts}
\end{figure}

\begin{figure}
\centering
\includegraphics{./fig2_mc_mass.png}
\caption{\textbf{Figure 2:} (a) Least square mean estimated mass
throughout the experiment and (b) Least square mean terminal brood mass.
Letters indicate significant differences both within and across temporal
treatment categories (constant and variable). Error bars represent 95\%
confidence intervals. Difference of 1 interval feeding day is equal to 3
Julian days}
\end{figure}

\begin{figure}
\centering
\includegraphics{./fig3_mc_growth.png}
\caption{\textbf{Figure 3:} (a) Least square mean estimated growth rate
by time periods relative to pulses and (b) Least square mean growth
rates by treatment. Letters indicate significant differences both within
and across temporal treatment categories (constant and variable), while
asterisks represent growth estimates significantly different than 0.
Error bars represent 95\% confidence intervals. Difference of 1 interval
feeding day is equal to 3 Julian days.}
\end{figure}

\begin{figure}
\centering
\includegraphics{./fig4_drones.png}
\caption{\textbf{Figure 4:} (a) Least square mean estimated cumulative
drone production and (b) Least square mean total drone production.
Letters indicate significant differences both within and across temporal
treatment categories (constant and variable). Error bars represent 95\%
confidence intervals. Difference of 1 interval feeding day is equal to 3
Julian days.}
\end{figure}

\begin{figure}
\centering
\includegraphics{./supfig1_pollen.png}
\caption{\textbf{Supplemental Figure 1:} (a,c) Least square mean
estimated cumulative pollen consumption (b,d) Least square mean total
pollen consumption by treatment. Letters indicate significant
differences both within and across temporal treatment categories
(constant and variable). Error bars represent 95\% confidence intervals.
Difference of 1 interval feeding day is equal to 3 Julian days.}
\end{figure}

\begin{figure}
\centering
\includegraphics{./supfig2_nectar.png}
\caption{\textbf{Supplemental Figure 2:} (a,c) Least square mean
estimated cumulative nectar consumption (b,d) Least square mean total
nectar consumption by treatment. Letters indicate significant
differences both within and across temporal treatment categories
(constant and variable). Error bars represent 95\% confidence intervals.
Difference of 1 interval feeding day is equal to 3 Julian days.}
\end{figure}

\hypertarget{references}{%
\section*{References}\label{references}}
\addcontentsline{toc}{section}{References}

\hypertarget{refs}{}
\leavevmode\hypertarget{ref-Benton2003}{}%
Benton, Tim G., Juliet A. Vickery, and Jeremy D. Wilson. 2003.
``Farmland biodiversity: Is habitat heterogeneity the key?''
\emph{Trends Ecol. Evol.} 18 (4): 182--88.
\url{https://doi.org/10.1016/S0169-5347(03)00011-9}.

\leavevmode\hypertarget{ref-Blaauw2014}{}%
Blaauw, Brett R., and Rufus Isaacs. 2014. ``Flower plantings increase
wild bee abundance and the pollination services provided to a
pollination-dependent crop.'' Edited by Yann Clough. \emph{J. Appl.
Ecol.}, April, n/a--n/a. \url{https://doi.org/10.1111/1365-2664.12257}.

\leavevmode\hypertarget{ref-Brown2016a}{}%
Brown, Mark J.F., Lynn V Dicks, Robert J Paxton, Katherine C.R. Baldock,
Andrew B Barron, Marie-pierre Chauzat, Breno M Freitas, et al. 2016. ``A
horizon scan of future threats and opportunities for pollinators and
pollination.'' \emph{PeerJ} 4 (3): e2249.
\url{https://doi.org/10.7717/peerj.2249}.

\leavevmode\hypertarget{ref-Cameron2011}{}%
Cameron, Sydney A, Jeffrey D Lozier, James P Strange, Jonathan B Koch,
Nils Cordes, Leellen F Solter, Terry L Griswold, and Gene E Robinson.
2011. ``Patterns of widespread decline in North American bumble bees.''
\emph{Proc. Natl. Acad. Sci. U. S. A.} 108 (2): 662--67.
\url{https://doi.org/10.1073/pnas.1014743108}.

\leavevmode\hypertarget{ref-Carvell2006b}{}%
Carvell, Claire, David B. Roy, Simon M. Smart, Richard F. Pywell, Chris
D. Preston, and Dave Goulson. 2006. ``Declines in forage availability
for bumblebees at a national scale.'' \emph{Biol. Conserv.} 132 (4):
481--89. \url{https://doi.org/10.1016/j.biocon.2006.05.008}.

\leavevmode\hypertarget{ref-Carvell2007}{}%
Carvell, C., W. R. Meek, R. F. Pywell, D. Goulson, and M. Nowakowski.
2007. ``Comparing the efficacy of agri-environment schemes to enhance
bumble bee abundance and diversity on arable field margins.'' \emph{J.
Appl. Ecol.} 44 (1): 29--40.
\url{https://doi.org/10.1111/j.1365-2664.2006.01249.x}.

\leavevmode\hypertarget{ref-Couvillon2010}{}%
Couvillon, M. J., and a. Dornhaus. 2010. ``Small worker bumble bees
(Bombus impatiens) are hardier against starvation than their larger
sisters.'' \emph{Insectes Soc.} 57: 193--97.
\url{https://doi.org/10.1007/s00040-010-0064-7}.

\leavevmode\hypertarget{ref-Dance2017}{}%
Dance, C., C. Botías, and D. Goulson. 2017. ``The combined effects of a
monotonous diet and exposure to thiamethoxam on the performance of
bumblebee micro-colonies.'' \emph{Ecotoxicol. Environ. Saf.} 139
(November 2016): 194--201.
\url{https://doi.org/10.1016/j.ecoenv.2017.01.041}.

\leavevmode\hypertarget{ref-Elliott2009}{}%
Elliott, Susan E. 2009. ``Surplus nectar available for subalpine bumble
bee colony growth.'' \emph{Environ. Entomol.} 38 (6): 1680--9.
\url{https://doi.org/10.1603/022.038.0621}.

\leavevmode\hypertarget{ref-Furst2014}{}%
Fürst, M a, D P McMahon, J L Osborne, R J Paxton, and M J F Brown. 2014.
``Disease associations between honeybees and bumblebees as a threat to
wild pollinators.'' \emph{Nature} 506 (7488): 364--66.
\url{https://doi.org/10.1038/nature12977}.

\leavevmode\hypertarget{ref-Goulson2008}{}%
Goulson, Dave. 2010. \emph{Bumble bees: behavior ecology, and
conservation}. 2nd ed. Oxford University Press.

\leavevmode\hypertarget{ref-Goulson2016}{}%
Goulson, Dave, and Elizabeth Nicholls. 2016. ``The canary in the
coalmine; bee declines as an indicator of environmental health.''
\emph{Sci. Prog.} 99 (3): 312--26.
\url{https://doi.org/10.3184/003685016X14685000479908}.

\leavevmode\hypertarget{ref-Goulson2015c}{}%
Goulson, Dave, Elizabeth Nicholls, C. Botias, Ellen L. Rotheray,
Cristina Botías, and Ellen L. Rotheray. 2015. ``Bee declines driven by
combined stress from parasites, pesticides, and lack of flowers.''
\emph{Science (80-. ).}, no. February: 1--16.
\url{https://doi.org/10.1126/science.1255957}.

\leavevmode\hypertarget{ref-Goulson2002c}{}%
Goulson, D., W. O.H. Hughes, L. C. Derwent, and J. C. Stout. 2002.
``Colony growth of the bumblebee, Bombus terrestris, in improved and
conventional agricultural and suburban habitats.'' \emph{Oecologia} 130
(2): 267--73. \url{https://doi.org/10.1007/s004420100803}.

\leavevmode\hypertarget{ref-Grab2019}{}%
Grab, Heather, Michael G Branstetter, Nolan Amon, Katherine R.
Urban-Mead, Mia G Park, Jason Gibbs, Eleanor J Blitzer, Katja Poveda,
Greg Loeb, and Bryan N Danforth. 2019. ``Agriculturally dominated
landscapes reduce bee phylogenetic diversity and pollination services.''
\emph{Science (80-. ).} 363 (6424): 282--84.
\url{https://doi.org/10.1126/science.aat6016}.

\leavevmode\hypertarget{ref-Greenleaf2007b}{}%
Greenleaf, Sarah S, Neal M Williams, Rachael Winfree, and Claire Kremen.
2007. ``Bee foraging ranges and their relationship to body size.''
\emph{Oecologia} 153 (3): 589--96.
\url{https://doi.org/10.1007/s00442-007-0752-9}.

\leavevmode\hypertarget{ref-Grixti2009}{}%
Grixti, Jennifer C., Lisa T. Wong, Sydney a. Cameron, and Colin Favret.
2009. ``Decline of bumble bees (Bombus) in the North American Midwest.''
\emph{Biol. Conserv.} 142 (1): 75--84.
\url{https://doi.org/10.1016/j.biocon.2008.09.027}.

\leavevmode\hypertarget{ref-Heinrich2004}{}%
Heinrich, Bernd. 2004. \emph{Bumblebee Economics}. Harvard University
Press.

\leavevmode\hypertarget{ref-Hines2005}{}%
Hines, Heather M, and Stephen D Hendrix. 2005. ``Bumble Bee (
Hymenoptera : Apidae ) Diversity and Abundance in Tallgrass Prairie
Patches : Effects of Local and Landscape Floral Resources.'' \emph{Env.
. Entomol} 34 (6): 1477--84.
\url{https://doi.org/10.1603/0046-225X-34.6.1477}.

\leavevmode\hypertarget{ref-Holzschuh2013}{}%
Holzschuh, Andrea, Carsten F. Dormann, Teja Tscharntke, and Ingolf
Steffan-Dewenter. 2013. ``Mass-flowering crops enhance wild bee
abundance.'' \emph{Oecologia} 172 (2): 477--84.
\url{https://doi.org/10.1007/s00442-012-2515-5}.

\leavevmode\hypertarget{ref-Jacobson2018a}{}%
Jacobson, Molly M., Erika M. Tucker, Minna E. Mathiasson, and Sandra M.
Rehan. 2018. ``Decline of bumble bees in northeastern North America,
with special focus on Bombus terricola.'' \emph{Biol. Conserv.} 217
(August 2017): 437--45.
\url{https://doi.org/10.1016/j.biocon.2017.11.026}.

\leavevmode\hypertarget{ref-Kerr2015}{}%
Kerr, Jeremy T, Alana Pindar, Paul Galpern, Laurence Packer, Simon G
Potts, Stuart M Roberts, Pierre Rasmont, et al. 2015. ``Climate change
impacts on bumblebees converge across continents.'' \emph{Science (80-.
).} 349 (6244): 177--80. \url{https://doi.org/10.1126/science.aaa7031}.

\leavevmode\hypertarget{ref-Landis2017}{}%
Landis, Douglas A. 2017. ``Designing agricultural landscapes for
biodiversity-based ecosystem services.'' \emph{Basic Appl. Ecol.} 18:
1--12. \url{https://doi.org/10.1016/j.baae.2016.07.005}.

\leavevmode\hypertarget{ref-Martins2018}{}%
Martins, Kyle T., Cécile H. Albert, Martin J. Lechowicz, and Andrew
Gonzalez. 2018. ``Complementary crops and landscape features sustain
wild bee communities.'' \emph{Ecol. Appl.} 28 (4): 1093--1105.
\url{https://doi.org/10.1002/eap.1713}.

\leavevmode\hypertarget{ref-McArt2017}{}%
McArt, Scott H., Christine Urbanowicz, Shaun McCoshum, Rebecca E. Irwin,
and Lynn S. Adler. 2017. ``Landscape predictors of pathogen prevalence
and range contractions in US bumblebees.'' \emph{Proc. R. Soc. B Biol.
Sci.} 284 (1867): 20172181.
\url{https://doi.org/10.1098/rspb.2017.2181}.

\leavevmode\hypertarget{ref-Moerman2017}{}%
Moerman, Romain, Maryse Vanderplanck, Denis Fournier, Anne Laure
Jacquemart, and Denis Michez. 2017. ``Pollen nutrients better explain
bumblebee colony development than pollen diversity.'' \emph{Insect
Conserv. Divers.} 10 (2): 171--79.
\url{https://doi.org/10.1111/icad.12213}.

\leavevmode\hypertarget{ref-Peat2005}{}%
Peat, J., J. Tucker, and Dave Goulson. 2005. ``Does intraspecific size
variation in bumblebees allow colonies to efficiently exploit different
flowers?'' \emph{Ecol. Entomol.} 30 (2): 176--81.
\url{https://doi.org/10.1111/j.0307-6946.2005.00676.x}.

\leavevmode\hypertarget{ref-Pelletier2003}{}%
Pelletier, Luc, and Jeremy N McNeil. 2003. ``The effect of food
supplementation on reproductive success in bumblebee field colonies.''
\emph{Oikos} 3 (3): 688--94.
\url{https://doi.org/10.1034/j.1600-0706.2003.12592.x}.

\leavevmode\hypertarget{ref-Rhemtulla2007}{}%
Rhemtulla, Jeanine M., David J. Mladenoff, and Murray K. Clayton. 2007.
``Regional land-cover conversion in the U.S. upper Midwest: Magnitude of
change and limited recovery (1850-1935-1993).'' \emph{Landsc. Ecol.} 22
(SUPPL. 1): 57--75. \url{https://doi.org/10.1007/s10980-007-9117-3}.

\leavevmode\hypertarget{ref-Rotheray2017}{}%
Rotheray, Ellen L, Juliet L Osborne, and Dave Goulson. 2017.
``Quantifying the food requirements and effects of food stress on bumble
bee colony development.'' \emph{J. Apic. Res.} 56 (3): 288--99.
\url{https://doi.org/10.1080/00218839.2017.1307712}.

\leavevmode\hypertarget{ref-Roulston2011}{}%
Roulston, T'ai H, and Karen Goodell. 2011. ``The role of resources and
risks in regulating wild bee populations.'' \emph{Annu. Rev. Entomol.}
56 (August): 293--312.
\url{https://doi.org/10.1146/annurev-ento-120709-144802}.

\leavevmode\hypertarget{ref-Rundlof2014}{}%
Rundlöf, Maj, Anna S. Persson, Henrik G. Smith, and Riccardo Bommarco.
2014. ``Late-season mass-flowering red clover increases bumble bee queen
and male densities.'' \emph{Biol. Conserv.} 172: 138--45.
\url{https://doi.org/10.1016/j.biocon.2014.02.027}.

\leavevmode\hypertarget{ref-Samuelson2018}{}%
Samuelson, Ash E, Richard J Gill, Mark J F Brown, and Ellouise
Leadbeater. 2018. ``Lower bumblebee colony reproductive success in
agricultural compared with urban environments.'' \emph{Proc. R. Soc. B
Biol. Sci.} 285 (1881): 20180807.
\url{https://doi.org/10.1098/rspb.2018.0807}.

\leavevmode\hypertarget{ref-Schellhorn2015c}{}%
Schellhorn, Nancy A., Vesna Gagic, and Riccardo Bommarco. 2015. ``Time
will tell: Resource continuity bolsters ecosystem services.''
\emph{Trends Ecol. Evol.} 30 (9): 524--30.
\url{https://doi.org/10.1016/j.tree.2015.06.007}.

\leavevmode\hypertarget{ref-Schmid-Hempel1998a}{}%
Schmid-Hempel, Regula, and Paul Schmid-Hempel. 1998. ``Colony
performance and immunocompetence of a social insect, Bombus terrestris,
in poor and variable environments.'' \emph{Funct. Ecol.} 12 (1): 22--30.
\url{https://doi.org/10.1046/j.1365-2435.1998.00153.x}.

\leavevmode\hypertarget{ref-Smith1998}{}%
Smith, Daryl D. 1998. ``Iowa prairie: Original extent and loss,
preservation and recovery attempts.'' \emph{J. Iowa Acad. Sci.} 105 (3):
94--108.

\leavevmode\hypertarget{ref-Spiesman2017}{}%
Spiesman, Brian J., Ashley Bennett, Rufus Isaacs, and Claudio Gratton.
2017. ``Bumble bee colony growth and reproduction depend on local flower
dominance and natural habitat area in the surrounding landscape.''
\emph{Biol. Conserv.} 206 (February): 217--23.
\url{https://doi.org/10.1016/j.biocon.2016.12.008}.

\leavevmode\hypertarget{ref-Tasei2008}{}%
Tasei, Jean-Noël, and Pierrick Aupinel. 2008. ``Validation of a method
using queenless Bombus terrestris micro-colonies for testing the
nutritive value of commercial pollen mixes by comparison with queenright
colonies.'' \emph{J. Econ. Entomol.} 101 (6): 1737--42.
\url{https://doi.org/10.1603/0022-0493-101.6.1737}.

\leavevmode\hypertarget{ref-Vaudo2018}{}%
Vaudo, Anthony D, Liam M Farrell, Harland M Patch, Christina M
Grozinger, and John F Tooker. 2018. ``Consistent pollen nutritional
intake drives bumble bee ( Bombus impatiens ) colony growth and
reproduction across different habitats.'' \emph{Ecol. Evol.} 8 (11):
5765--76. \url{https://doi.org/10.1002/ece3.4115}.

\leavevmode\hypertarget{ref-Westphal2009a}{}%
Westphal, C., I. Steffan-Dewenter, and T. Tscharntke. 2009. ``Mass
flowering oilseed rape improves early colony growth but not sexual
reproduction of b1. Westphal, C., Steffan-Dewenter, I. \& Tscharntke, T.
(2009). Mass flowering oilseed rape improves early colony growth but not
sexual reproduction of bumblebees. J.'' \emph{J. Appl. Ecol.} 46 (1):
187--93. \url{https://doi.org/10.1111/j.1365-2664.2008.01580.x}.

\leavevmode\hypertarget{ref-Williams2012b}{}%
Williams, Neal M., James Regetz, and Claire Kremen. 2012.
``Landscape-scale resources promote colony growth but not reproductive
performance of bumble bees.'' \emph{Ecology} 93 (5): 1049--58.
\url{https://doi.org/10.1890/11-1006.1}.

\leavevmode\hypertarget{ref-Woodard2017}{}%
Woodard, S. Hollis, and Shalene Jha. 2017. ``Wild bee nutritional
ecology: predicting pollinator population dynamics, movement, and
services from floral resources.'' \emph{Curr. Opin. Insect Sci.} 21
(Figure 1): 83--90. \url{https://doi.org/10.1016/j.cois.2017.05.011}.


\end{document}
